\documentclass{article}
\usepackage[utf8]{inputenc}
\title{Estructures Computadors}
\author{Sistach Reinoso, Arnau}

% nomes crec que serveix per la ela geminada (on sense el corrector dubto mai fer-ho anar 183 ·)
\usepackage[catalan]{babel}

% perque quedi mes clar
\usepackage{color}

% fer el rcases
\usepackage{mathtools}

% faig anar equation*
\usepackage{amsmath}


% declarant funcions
\DeclareMathOperator{\Ima}{Im}
\DeclareMathOperator{\SL}{SL}
\DeclareMathOperator{\Id}{Id}

\usepackage{amssymb}
\newcommand{\N}{\mathbb{N}}
\newcommand{\Z}{\mathbb{Z}}
\newcommand{\Q}{\mathbb{Q}}
\newcommand{\R}{\mathbb{R}}
\newcommand{\C}{\mathbb{C}}




% like kindle

\usepackage{microtype}     % microtypography, reduces hyphenation

\usepackage[papersize={3.6in,4.8in},hmargin=0.1in,vmargin={0.1in,0.1in}]{geometry}  % page geometry

\usepackage{fancyhdr}   % headers and footers
\pagestyle{fancy}
\fancyhead{}            % clear page header
\fancyfoot{}            % clear page footer

\setlength{\abovecaptionskip}{2pt} % space above captions 
\setlength{\belowcaptionskip}{0pt} % space below captions
\setlength{\textfloatsep}{2pt}     % space between last top float or first bottom float and the text
\setlength{\floatsep}{2pt}         % space left between floats
\setlength{\intextsep}{2pt}        % space left on top and bottom of an in-text float

\begin{document}
\maketitle
\tableofcontents
\newpage
\section{Elements bàsics}
\begin{itemize}
\item CPU - Central Processing Unit
	\begin{itemize}
	\item UC - Unity Control
		\subitem IR - Instruction Registrer
	\item UP - Uniti Proces
	\end{itemize}
\end{itemize}
Vida d'una instrucció
\begin{itemize}
\item IF - Instruction Fetch
\item OF - Operand Fetch
\item EX - Execute
\item WB - Write Back
\end{itemize}
Més definicioins
\begin{itemize}
\item CPI - Cicles de rellotge Per Instrucció
\item MIPS - Millons d'Instruccions Per Segon
	\begin{itemize}
	\item $f_c[MHz]$ - freqüència de relotge en MHz
	\item MIPS = $\frac{f_c[MHz]}{\text{CPI}}$
	\end{itemize}
\end{itemize}
Instruccions - Exemple simple
\begin{itemize}
\item Registre
	\begin{itemize}
	\item Aritmètiques
	\item Lògic
	\item Moure
	\item Desplaçar
	\end{itemize}
\item Memòria
	\begin{itemize}
	\item Load
	\item Store
	\end{itemize}
\item Control
	\begin{itemize}
	\item Jump
	\item Branch ``cada una al seu canto''
	\end{itemize}
\item Misce\lgem ània
	\begin{itemize}
	\item Noop
	\item Altres
	\end{itemize}
\end{itemize}
Pipeline - RISC
\begin{itemize}
\item CPI $\to$ 1
	\begin{itemize}
	\item $S$ = $\frac{T_\text{no pipeline}}{T_\text{pipeline}}$ Guany
	\item $m$ és el nombre d'instruccions
	\item $E = \frac{S}{m}$ Eficiència
	\end{itemize}
\item Necessitat de 2 memòries independents
	\begin{itemize}
	\item codi
	\item dades
	\end{itemize}
\item Problemes
	\begin{itemize}
	\item Data, Noop + reordenar = static
	\item Data Forwarding - Dinàmic
		\begin{itemize}
		\item Solució devant de dependències de dades
		\item 2 nous camins
		\end{itemize}
	\item Branch
		\begin{itemize}
		\item 
		\end{itemize}
	\end{itemize}
\end{itemize}

\newpage
\section{Futur pròxim}
\begin{itemize}
\item pentium Pro
\item power pc 620
\item ultrasparc I
\item alpha 21164a
\item MIPS R10000
\item HP-PA 8000 ``més m'agrada''
\end{itemize}
\subsection{Trouvé}
Similar a la catché, però aquesta tindrà el seu propi codi. Per a funcionar, necessitarà d'esdeveniments.

Capacitat de crear codi orientat a esdeveniments per evitar un màxim condicions propies de codi. Que aquest pugui fer feina sense preocupar-se gaire.
\subsubsection{Data forwarding}
Ve a ser molt similar a la catché, però ara a nivell del CPU.

Només vindrie a ser un sistema de programar la CPU, per a no necessitar d'una solució dinàmica. El mateix, però el mateix codi se faria responsable.

Si al final sembla que tot se simplifica menys la UC jajaja.

\subsection{Compilador}
Pel problema de trobar una solució de l'equació quadrada. Doncs eso.

Que pugui veure quina es la forma amb menys instruccions possibles :)

\end{document}

Imprimir la pagina 10 2-01, estaria prou be ;)
