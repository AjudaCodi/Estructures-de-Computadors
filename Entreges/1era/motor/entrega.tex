\documentclass{article}
\usepackage[utf8]{inputenc}
\usepackage[catalan]{babel}
\usepackage{amsmath}

\title{Entrega Pipeline}
\author{Sistach Reinoso, Arnau}

\begin{document}
\maketitle
\newpage
\tableofcontents
\newpage
\section{Enunciat}
Tenim un problema amb ell, ja que només tenim accés a una x. I tenim dues solucions. Així que me inventat x$_1$ i x$_2$ per a poder mostrar totes les solucions.\\
Em suposat que R8 té l'adreçament base de memòria.\\
No em guardat al mateix registre que operem.\\
Em fet la comprovació per a saber si te solució real.\\
Em usat Data Forwarding Paths, cosa que ha fet que no aparexeix cap Noop, molt agradable.\\
Em usat predicció estàtica de salt quan predeix que no es compleix.

\section{Guany}
Són 12 operacions diferents.\\
En Von Newman tardaria 12*4 = 48 tc.\\
El nostre tarda un total de 15 tc.\\
$$\text{Guany} = \frac{48}{15} = 3.2$$

\section{Taules}
\end{document}
